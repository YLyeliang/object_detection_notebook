\documentclass{article}

\usepackage[UTF8]{ctex}
\usepackage{amsmath}
\usepackage{graphicx}


\title{目标检测相关细节}
\author{叶亮}
\date{\today}
\begin{document} 
\maketitle
\section{Loss}
\subsection{IoU loss}


\subsection{GIoU loss}

\subsection{DIoU loss}

Code optimization, the calculation of distance between two center points of bboxes.
\begin{equation}
\begin{aligned}
(C_1-C_2)^2)&= ((x1_{C1}+x2_{C1})/2-(x1_{C2}+x2_{C2})/2)^2 \\
&=((x1_{C1}+x2_{C1})-(x1_{C2}+x2_{C2}))^2/4 \\
\end{aligned}
\end{equation}
\subsection{CIoU loss}
\subsection{BIoU loss}

\section{bbox \& anchor}

\subsection{coder}
在RetinaNet、SSD、Cascade R-CNN等网络中,网络预测的bbox都会进行Delta xywh编码,即将原始的(xmin, ymin, xmax, ymax)进行编码,计算相对距离,来减小回归的过拟合,提升回归的稳定性。
\begin{equation}\textbf{•}
\begin{aligned}
&\delta_x = (g_x-p_x)/p_w	   & \delta_y = (g_y - p_y)/p_h \\
&\delta_w = \log{g_w/p_w}	&\delta_h = \log{g_h/p_h}
\end{aligned}\label{bbox_coder}
\end{equation}

上述公式计算得到的$\delta$值通常很小,因为网络通常只对p进行少量微调,导致回归loss比分类loss小很多。为了提升学习的有效性,$\delta$通常需要经过均值和方差进行标准化.

\begin{equation}
\begin{aligned}
\delta_{x}^{'}=\frac{\delta_x-\mu_x}{\rho_x}
\end{aligned}
\end{equation}

\section{training}
\subsection{Optimizer}
\subsubsection{warmup}
warmup通常有三个方式:linear, constant, exp. 通常需要设置warmup的迭代数$iter_{total}$和warmup的增加比率ratio,
\begin{align}
lr_t = lr_{constant}*ratio
\end{align}
\begin{equation}
\begin{aligned}
lr_t = lr_{const}*(1-k), \\
k= (1-iter_t /iter_{total}) * (1-ratio)
\end{aligned}
\end{equation}
\begin{align}
lr_t = lr_{const}*k, \\
k=ratio^{1-iter_t/iter_{total}}
\end{align}

\section{Yolov4}
Yolov4的实现过程中,anchor的生成以及label与anchor的对应关系的构建方法。
\subsection{Anchor generation \& target build}
与retinanet,rcnn系列等bbox预测不同的是,yolo模型推理得到的结果(x,y,w,h)需要经过如下公式进行编码:
\begin{equation}
\begin{aligned}
b_x=\rho(t_x)+c_x
\end{aligned}
\end{equation}
\begin{figure}
\centering
\includegraphics[scale=0.5]{images/yolo_bbox.png}
\caption{Yolo bbox prediction}
\end{figure}

\section{backbone}
\subsection{Regularization}
\subsubsection{Dropout}
原理,dropout随机丢弃神经元(全连接中输入神经元),实现方式为$keep_prob$,每个神经元生成一个随机数$k,k<keep_prob$即丢弃。优点,该方法有利于分类中泛化能力的提升.
\subsubsection{Drop Connect}
\subsubsection{Drop block}

\section{Refinedet}
\subsection{Anchor}
Refine中的anchor计算。对于每一个feature map, 首先计算其mesh grid,然后计算每个框的中心点$(x,y)=((i+0.5)/featsize,(j+0.5)/featsize)$,然后根据每个feature map对应的anchor box的大小,计算anchor的长和宽$WH_{ki}=box_k/imagesize, where k means kth feature level, i means ith grid$. e.g, 使用四个feature level, $box=[32,64,128,256], imagesize=320, featsize=[40,20,10,5], aspect ratio=[2,2,2,2]$, 那么最终生成$40*40*3+20*20*3+10*10*3+5*5*3=6375$个anchor.


\section{Dataset}
\subsection{COCO Dataset}
COCO dataset 全称 Common Objects in Context. 共有80个类。共有三种标注类型:object instance, object keypoints, image captions。使用json文件进行存储。
\subsubsection{object instance}
{
    "info": info,
    "licenses": [license],
    "images": [image],
    "annotations": [annotation],
    "categories": [category]
}

\section{Detector}
\subsection{CornerNet, CenterNet}
\subsubsection{targets computation}
在anchor free网络中,生成gt bbox的heatmap target时,相应的高斯半径的计算方式如下。一共存在三种情况:
1,生成的target与gt bbox有重叠部分,其中一个corner在gt box内部,另一个在gt box外部。
\begin{align}
        \cfrac{(w-r)*(h-r)}{w*h+(w+h)r-r^2} \ge {iou} \quad\Rightarrow\quad
        {r^2-(w+h)r+\cfrac{1-iou}{1+iou}*w*h} \ge 0 \\
        {a} = 1,\quad{b} = {-(w+h)},\quad{c} = {\cfrac{1-iou}{1+iou}*w*h} \\
        {r} \le \cfrac{-b-\sqrt{b^2-4*a*c}}{2*a}
\end{align}
2.两个corner都在gt box里面
\begin{align}
\cfrac{(w-2*r)*(h-2*r)}{w*h} \ge {iou} \quad\Rightarrow\quad
        {4r^2-2(w+h)r+(1-iou)*w*h} \ge 0 \\
        {a} = 4,\quad {b} = {-2(w+h)},\quad {c} = {(1-iou)*w*h} \\
        {r} \le \cfrac{-b-\sqrt{b^2-4*a*c}}{2*a}
\end{align}
3.两个corner都在gt box外部
\begin{align}
\cfrac{w*h}{(w+2*r)*(h+2*r)} \ge {iou} \quad\Rightarrow\quad
        {4*iou*r^2+2*iou*(w+h)r+(iou-1)*w*h} \le 0 \\
        {a} = {4*iou},\quad {b} = {2*iou*(w+h)},\quad {c} = {(iou-1)*w*h} \\
        {r} \le \cfrac{-b+\sqrt{b^2-4*a*c}}{2*a}
\end{align}
高斯核的计算
\begin{align}
\exp^{\cfrac{-(x*x+y*y)}{2*\sigma*\sigma}}
\end{align}

\subsubsection{decode heatmap}
对heatmap进行解码,遵循如下顺序:1. nms on heatmap. 2. get topk positions from heatmap. 3. decode offset and wh size.


\end{document}